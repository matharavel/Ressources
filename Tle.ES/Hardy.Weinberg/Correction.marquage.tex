\documentclass[xcolor={dvips,ps2pdf},dvips]{beamer}
 \usepackage{hyperref}
 \usepackage[greek,frenchb]{babel}
%\usepackage[utf8x]{inputenc}
\usepackage{ucs}
\usepackage[latin1]{inputenc}
\usepackage{lmodern}
\usepackage{tikz,tkz-tab}
\usefonttheme{serif,professionalfonts}
\usepackage{amsfonts}
\usepackage{mathtools}
\usepackage{pst-text,pst-plot,pst-math}
\usepackage{mathrsfs}
\usepackage{color}
\definecolor{myred}{rgb}{0.93,0.26,0}
\usepackage{systeme}
%\usepackage{colortbl}
\usepackage{multicol}
\newcommand{\R}{\mathbb{R}}
\newcommand{\N}{\mathbb{N}}
\newcommand{\Z}{\mathbb{Z}}
\newcommand{\Q}{\mathbb{Q}}
\newcommand{\D}{\mathbb{D}}\newcommand{\vect}[1]{\overrightarrow{\,\mathstrut#1\,}}
\newsavebox{\fmbox}

\newenvironment{encadrer}{%
	\vspace{1em}
    \begin{lrbox}{\fmbox}\begin{minipage}{0.98\textwidth}}
    {%\vspace{1em}
    \end{minipage}\end{lrbox}\fbox{\usebox{\fmbox}}
	
	\vspace{1em}
	}
\everymath{\displaystyle}
\usepackage{tabularx}
\usepackage{pifont}
\usepackage{ifthen}
\usepackage{amsmath,multicol}
\usepackage{fancybox,color}
\usepackage{xcolor}
\usepackage[pstricks]{bclogo}
\usepackage{pstricks,pst-node,pst-text,pst-coil}
\setcounter{tocdepth}{1} %\setcounter{page}{0}
%\setcounter{page}{0}
\setbeamertemplate{section in toc}[sections numbered]
\newcommand{\pr}{\mathbb{P}}
%\renewcommand\FancyVerbFormatLine[1]{\colorbox{green}{#1}}

%\usecolortheme[rgb=0.97,0.35,0.04]{structure}
\usepackage{caption}
\usecolortheme{dolphin}
\useoutertheme{shadow}
\captionsetup{labelformat=empty,font=footnotesize}


\usetheme{Warsaw}
%\usecolortheme{seahorse}
\setbeamertemplate{blocks}[rounded][shadow=true]
\setbeamertemplate{items}[ball]
%\setbeamertemplate{navigation symbols}{}
\setbeamercolor{title}{bg=red!65!black, fg=white}



\begin{document}


\title[] 
\subtitle{}

\author[] % (optional, use only with lots of authors)
{}
% - Give the names in the same order as the appear in the paper.
% - Use the inst{?} command only if the authors have different
%   affiliation.

\institute{Lyc�e Maurice Ravel}% (optional, but mostly needed)

\logo{}




% If you wish to uncover everything in a step-wise fashion, uncomment
% the following command: 

\beamerdefaultoverlayspecification{<+->}




\everymath{\displaystyle}


\begin{frame}
\setbeamertemplate{blocks}[rounded][shadow=true]
\setbeamertemplate{items}[ball]
%\setbeamertemplate{navigation symbols}{}
\setbeamercolor{title}{bg=red!65!black, fg=white}

\frametitle{Exercice 1}
\pause 
on a $\dfrac{M}{N}=\dfrac{m}{n}$ donc $N=\dfrac{82\times 67}{2}=\pause 2747$.

On peut estimer le nombre d'animaux en 2019 � $2747$ :~\pause 

Ainsi, la population de Trichosurus cunninghami a retrouv� un niveau sensiblement �quivalent
� celui d'avant 2009.

\end{frame}

\begin{frame}
\frametitle{Exercice 2}
\pause 
\begin{enumerate}
\item 
Ici, les otaries sont marqu�es en coupant une m�che de fourrure. Cette m�thode a plusieurs
avantages :
\pause
\begin{itemize}
\item sans un temps assez court, elle est ind�l�bile et, dans un temps long, elle dispara�t
(avec la repousse des poils) ce qui ne perturbe pas la vie de l'animal sur le long terme ;
\pause 
\item elle permet une recapture visuelle c'est-�-dire par simple observation des animaux (et
donc sans avoir � la recapturer physiquement)
\end{itemize}
\pause 
\item 
\begin{itemize}
\item Pour la capture 1 : \pause  on peut estimer l'abondance � $\dfrac{1291\times 1080}{391}\approx 3566$ individus.
\pause 
\item Pour la capture 2 : \pause  on peut estimer l'abondance � $\dfrac{1291\times 1224}{378}\approx 4180$ individus.
\pause 
\item 
Pour la capture 3 : \pause  on peut estimer l'abondance � $\dfrac{1291\times 1233}{357}\approx 4459$ individus.
\end{itemize}
\end{enumerate}
\end{frame}
\begin{frame}
\begin{enumerate}
\item[3.] L'abondance moyenne (estim�e) est environ � : \pause 
\begin{center}
$\dfrac{3566 + 4180 + 3937 + 4459}{4}=4036$.
\end{center}
\pause
\item[4.]  On voit que les estimations diff�rent selon les captures en raison de la fluctuation
d'�chantillonnage (le r�sultat obtenu varie en fonction de l'�chantillon). R�aliser plusieurs
recaptures permet de faire une moyenne sur plusieurs estimations ce qui permet de lisser
les �carts dus � la fluctuation d'�chantillonnage.
\end{enumerate}


\end{frame}



\begin{frame}
\frametitle{Exercice 4}
\pause 
\begin{enumerate}
\item Il y a 4 buissons de chaque type dans chacune des deux zones donc \pause $4\times 2\times 2= 16$  buissons.\\
\pause Sur chaque buisson, on a rel�ch� 24 animaux soit \pause $16\times 24=384$ \pause  animaux au total.
\pause 
\item Ici, on marque tous les animaux qu'on �tudie et on recapture tous les animaux sauf ceux
qui sont morts. \\
\pause Le � marquage � alors est le fait de ne peut pas �tre recaptur�.\\
\pause 
\item Chaque lot comprend autant d'individus m�les que femelles pour �viter les biais li�s
au sexe. \\
\pause Le marquage individuel � l'aide d'un marqueur permanent � pointe fine sur
l'abdomen permet d'avoir une marque ind�l�bile mais qui n'affecte pas le comportement
de l'animal ni son esp�rance de vie. \\
\pause 
Le temps entre le marquage et la r�introduction des animaux dans le milieu est cours (5 jours) et les recaptures sont faites � des temps suffisamment courts (3, 10, 17 et 24 jours) pour limiter les modifications de la population
(naissance, morts naturelles...).
\end{enumerate}
\end{frame}
  \end{document}